\clearpage
\section{Processo de Fabricação}

\subsection{Moagem do Malte}

O malte é enviado para os moinhos que possuem como função promover um corte na casca do grão fazendo com que ela permaneça o mais íntegra possível, e então liberar o amido para o processo. A moagem também promove a diminuição do tamanho da partícula do amido, ocasionando um aumento na velocidade de hidrólise do amido.

O objetivo de manter a casca do grão intacta é que ela servirá de filtro no final do processo de mostura. A massa de grãos formado no fundo da panela permite que o mosto de cevada seja filtrado e levado para a fervura mais limpo. Cerveja ficará mais límpida, o rendimento da produção é maior por ter menos descarte de sujeira ao final da fervura.

\subsection{Brassagem}
Durante a brassagem o malte é hidratado e as enzimas presentes nele são ativadas. O amido pode ser convertido em açúcares fermentáveis e as proteínas convertidas em nutrientes. Porém, as enzimas necessárias para a conversão somente são ativadas diante de temperaturas pré-determinadas.

Os maltes moídos são colocados na panela de mostura e misturados com água aquecida à temperatura de 70\textdegree C (não deixar formar grumos), a temperatura abaixa naturalmente para 66\textdegree C, devesse cessar o aquecimento e deixar o mostura em repouso por 60 minutos com a panela tampada, durante esse período se a temperatura baixar para menos de 64\textdegree C devesse aquecer novamente e mexer até atingir 66 \textdegree C. Toda vez que se aquecer a mistura devesse agitar, para obter uma leitura mais precisa no termômetro. Atingindo a temperatura, é importante que a temperatura não ultrapasse 72oC, pois em altas temperaturas as enzimas são inativadas. As enzimas contidas no malte são liberadas para o meio e sob ação de calor são ativadas para promover a hidrólise catalítica do amido.

Antes de completar os 60 minutos deve-se fazer o teste de iodo, pingar algumas gotas da mostura sobre um azulejo branco ou prato de porcelana também branco. Após pegue o frasco do iodo 2\% e pingue 1 gota sobre o mostura que foi colocado no azulejo. Se a coloração for amarela ouro, após os 60 minutos, prosseguir com a receita aquecendo até 75\textdegree C. Se ainda estiver apresentando vestígios ou até coloração forte de uma “cor marrom escuro” a mostura deverá ficar mais alguns minutos na temperatura de 66\textdegree C.
Após estes 60 minutos elevar a temperatura para 75\textdegree C em 5 minutos sempre agitando. A temperatura não deve subir bruscamente. Se após este tempo permanecer a cor marrom escuro interromper o processo. As causas desse problema podem ser:

\begin{itemize}
    \item Falhas na moagem do malte.
    \item Termômetro descalibrado.
    \item Temperatura da mostura subiu no início da mostura, acima dos 75\textdegree C o que desativou as enzimas.
\end{itemize}

Em 75\textdegree C deixar descansando mais 10 minutos com a panela tampada.

O objetivo da mosturação é otimizar o rendimento de extração, ter produtividade máxima (n\textdegree  fabricações/dia) e custos operacionais mínimos (energia e pessoal).

\subsubsection{Fatores que afetam a ação enzimática}

\subsubsection{A concentração da mostura}

A degradação enzimática do amido é feita na mosturação, o amido pode ser degradado pelas amilases:

\begin{itemize}
    \item \(\alpha\)-amilase: endoenzima e
    \item \(\beta\)-amilase: exoenzima.
\end{itemize}

Não pode haver ação enzimática sem água, a proporção enzima/substrato influi no resultado da ação das enzimas. Quanto menor a proporção água: malte, melhor a ação de proteases e \(\beta\)-amilase e pior a ação de \(\alpha\)-amilase

\subsubsection{Tempo de Mostura}

A mosturação demora cerca de 60 minutos, podendo ser maior ou menor depende da quantidade de amido consumido nesse período.

A atividade enzimática depende da temperatura a qual o mostura está sendo mantido, No início do processo a atividade enzimática é maior devido à alta concentração de amido.

\subsubsection{Temperatura de Mostura}
É necessário controlar a temperatura pois em altas temperaturas as enzimas são inativadas.


\subsubsection{pH da mostura}

Condições ótimas das enzimas:

\begin{table}[h]
\centering
\caption{Condições ótimas das enzimas}
\begin{tabular}{|l|c|c|l|}
\hline
\multicolumn{1}{|c|}{{\bf Enzima}} & {\bf \begin{tabular}[c]{@{}c@{}}Faixa ideal de\\ temperatura\end{tabular}} & {\bf Faixa de pH} & \multicolumn{1}{c|}{{\bf Função da enzima}} \\ \hline
Phytase & 30 - 52\textdegree C & 5.0 - 5.5 & Diminuição do pH da mostura \\ \hline
Debranching (var.) & 35 - 45\textdegree  C & 5.0 - 5.8 & Solubilização de Amidos \\ \hline
Beta Glucanase & 35 - 45\textdegree  C & 4.5 - 5.5 & \begin{tabular}[c]{@{}l@{}}Gelatinização, auxiliando na liberação\\ de açúcares disponíveis\end{tabular} \\ \hline
Peptidase & 45 - 55\textdegree C & 4.6 - 5.3 & \begin{tabular}[c]{@{}l@{}}Produz maior quantidade de proteínas\\ solúveis no mosto\end{tabular} \\ \hline
Protease & 45 - 55\textdegree C & 4.6 - 5.3 & \begin{tabular}[c]{@{}l@{}}Quebra proteínas que geram turvação\\ da cerveja\end{tabular} \\ \hline
\(\beta\)-amilase & 55 - 65\textdegree C & 5.0 - 5.5 & Produz maltose \\ \hline
\(\alpha\)-amilase & 68 - 72\textdegree C & 5.3 - 5.7 & \begin{tabular}[c]{@{}l@{}}Produz açúcares diversos, incluindo\\ maltose\end{tabular} \\ \hline
\end{tabular}
\end{table}

\subsubsection{Qualidade do malte}
Quanto melhores dissoluções citolítica/ proteolítica, menor será o tempo de processo (produtividade maior) e o Custo operacional.

\subsection{Recirculação}

Pode ser feita manualmente abra-se a torneira devagar e despeje o conteúdo em recipiente. Devolva o mosto a panela utilizando uma escumadeira. O “chuveiro” evitará que o liquido remexa o sedimento de malte que se formará no fundo. Repita o processo até não haver mais partículas em suspensão.

\subsection{Filtração}

A filtração consiste na separação do mosto (parcela líquida) do bagaço (parcela sólida). O objetivo desta separação é a obtenção do máximo em extrato do malte sacarificado.

Um mosto clarificado é uma condição necessária para se obter uma cerveja de boa qualidade.

Há várias maneiras de filtrar a mostura, passar a mostura (malte + água) através de um saco branco duplo. Este saco deve estar bem limpo e esterilizado com água quente, poderá se utilizar uma panela com um fundo falso, deve-se colocar água a 75\textdegree C no fundo da panela até começar a sair pelos furos do fundo falso.

Para facilitar a extração do açúcar residual contido ainda no bolo de bagaço, adicione água a 75\textdegree C e depois misture com o mosto já filtrado.

No caso do uso de um saco branco após filtrado o primeiro mosto, adiciona-se água a 75\textdegree C em uma só vez. Misturar todo o mosto contido na panela de fervura que deverá ficar com densidade de 1.044.

\subsection{Fervura e Lupagem}

A fervura inicia-se apóa coleta de todo mosto e transferência a outro recipiente. 
O processo da fervura é necessário para desenvolvimento de sabores e coagulação de proteínas, assim, a etapa, em média, dura no mínimo 60 minutos. Durante o processo, ocorre também a adição de lúpulo que influencia em três características a cerveja: amargor, aroma e sabor. A determinação destas características depende do tempo da fervura em que o lúpulo é adicionado, uma vez que a extração de suas propriedades varia de acordo com o tempo e que ocorre a fervura. Além destas características, o lúpulo atua na estabilidade da espuma e como conservante natural.

De forma simples e geral, a adição de lúpulo segue as seguintes fases (em fervura de tempo mínimo de 60 minutos):

\begin{itemize}
    \item Para maior obtenção de amargor, o lúpulo é adicionado no início da fervura;
    \item Para maior obtenção de sabor, o lúpulo é adicionado na metade do tempo de fervura;
    \item Para maior obtenção de aroma, o lúpulo é adicionado ao final da fervura, ou quando esta já se encerrou.
\end{itemize}

Ressalta-se que o lúpulo confere as medidas de amargos, sabor e aroma.

Para adicionar o lúpulo, normalmente conta-se a adição com base no tempo que resta para terminar fervura

Para contagem de tempo, observa-se o início de fervura de maneira vigorosa.

\subsection{Resfriamento}

O resfriamento do mosto pode ser considerado como uma das etapas críticas no processo de produção, uma vez que é elevada a possibilidade de contaminação.

Ao resfriar, a grande quantidade de açúcares presentes no mosto o torna um excelente meio de cultura para as leveduras que serão adicionadas para fermentar a cerveja, bem como as leveduras selvagens que estão presentes no ar e nos equipamentos utilizados. Por isso, à medida que o mosto é resfriado é necessário tomar cuidado para minimizar o risco de contaminação.

A etapa de resfriamento deve acontecer o mais rapidamente possível. Quanto menor o tempo de resfriamento, melhor. Isto porque o quanto antes o mosto atingir a temperatura de fermentação, mais rápido ocorre a transferência para o fermentador e consequente diminuição das chances de contaminação.

Ao deixar simplesmente a panela tampada e esperar o mosto esfriar em temperatura ambiente, ocorrerá formação de sabores indesejáveis na cerveja, pois em alta temperatura e com baixa taxa de resfriamento, compostos diferentes serão gerados e não evaporados, que acarretará em sabores indesejáveis.

É importante também não mexer no mosto enquanto ocorre resfriamento.

A temperatura máxima ideal para fim de resfriamento é de 30\textdegree C. Esta temperatura evitará a inutilização do fermento ou que ocorra início de fermentação muito vigoroso.

\subsection{Oxigenação}
 A oxigenação do mosto é um fator importante para as leveduras e obtenção de uma fermentação adequada. O oxigênio contido no mosto é absorvido pelo fermento dentro das primeiras duas horas de inoculação e é o único momento em que a oxigenação é necessária. O fermento utiliza o oxigênio como nutriente e também para se reproduzir nas primeiras horas de fermentação. O nível de oxigênio recomendado no mosto é de 8 a 12 ppm.

 \subsection{Inoculação do fermento}
 
A etapa de inoculação deve ser feita logo após a oxigenação do mosto e varia de acordo com o tipo de fermento, se é seco ou líquido. Uma inoculação insuficiente leveduras poderá ocasionar um retardo muito grande para o início da fermentação, aumentando assim, a competição pelo mosto por bactérias e leveduras selvagens, levando a alta formação de ésteres frutados, acetaldeido e diacetil.
 
A vedação do recipiente fermentador impede a entrada de ar externo com contaminantes. O air-lock atua como mecanismo que permite a troca de gás conforme a relação da pressão interna com a externa. O motivo de usar fluído sanitizante no air-lock é a de impedir que haja contaminação por refluxo deste fluído para o interior do recipiente fermentador ou pelo retorno de ar externo (pressão externa maior que interna) para o interior do recipiente fermentador.

Um das etapas de maior importância dentro da produção de cerveja é a fermentação, que consiste na obtenção de energia (açúcares) por parte das leveduras (fungos) produzindo etanol e CO2.

Existem diversos tipos de leveduras, cerca de 700 espécies, porém para serem utilizadas na produção da cerveja devem possuir alto grau de consumo de açúcares fermentescíveis, (glicose, maltose e maltotriose) para garantirem a produção de etanol e subprodutos que conferem gostos e aromas desejáveis. Além disso as leveduras devem ser adaptadas a PH baixo para redução do risco de contaminação, tendo em vista que ambientes mais ácidos são propícios a proliferação de bactérias.

Na produção de cerveja existem dois tipos de leveduras: as de alta fermentação (ALE) e as de baixa fermentação (LAGERS). Na produção da Ale, a fermentação é realizada na parte superior do tanque de armazenamento, com temperaturas mais altas, em torno de 19 e 24\textdegree C. Já na Lager, o processo ocorre em temperaturas mais baixas, entre 8 e 12\textdegree C, fazendo com que a fermentação aconteça perto do fundo do tanque de fermentação. Portanto, para a inoculação do fermento é necessário conhecer a levedura utilizada pois o processo difere para cada um dos casos.

Antes de acrescentar a levedura a densidade do mosto deve ser medida para análise do rendimento da brassagem e para o cálculo de quantidade de células necessárias no processo. Esse cálculo é realizado segundo o livro “\textit{An Analysis of Brewing Techniques}” em que:

\newpage
\begin{table}[h]
\centering
\caption{Inoculação de fermentos secos p/ 20 litros de mosto}
\begin{tabular}{|c|c|c|c|c|c|c|c|}
\hline
\multicolumn{2}{|c|}{\begin{tabular}[c]{@{}c@{}}Densidade\\ do Mosto\end{tabular}} & \multicolumn{2}{c|}{\begin{tabular}[c]{@{}c@{}}Células viáveis \\ necessárias\\ (bilhões)\end{tabular}} & \multicolumn{2}{c|}{\begin{tabular}[c]{@{}c@{}}Número de sachets,\\ ALES/Condição\\ do Fermento\end{tabular}} & \multicolumn{2}{c|}{\begin{tabular}[c]{@{}c@{}}Número de sachets,\\ LAGERS/Condição\\  do Fermento\end{tabular}} \\ \hline
\begin{tabular}[c]{@{}c@{}}Densidade\\ Absoluta\end{tabular} & \begin{tabular}[c]{@{}c@{}}Densidade\\ (graus plato)\end{tabular} & Ales & Lagers & Péssima & Boa & Péssima & Boa \\ \hline
1032 & 8 & 120 & 240 & 1 & 0,5 & 3 & 1 \\ \hline
1036 & 9 & 135 & 270 & 2 & 0,5 & 3 & 1 \\ \hline
1040 & 10 & 150 & 300 & 2 & 0,5 & 4 & 1 \\ \hline
1044 & 10,9 & 165 & 330 & 2 & 1 & 4 & 1,5 \\ \hline
1048 & 11,9 & 178 & 356 & 2 & 1 & 5 & 1,5 \\ \hline
1052 & 12,8 & 192 & 384 & 2 & 1 & 5 & 1,5 \\ \hline
1056 & 13,7 & 205 & 410 & 3 & 1 & 5 & 1,5 \\ \hline
1060 & 14,6 & 219 & 438 & 3 & 1 & 6 & 2 \\ \hline
1064 & 15,6 & 234 & 468 & 4 & 1 & 6 & 2 \\ \hline
1068 & 16,5 & 247 & 494 & 4 & 1 & 7 & 2 \\ \hline
1072 & 17,4 & 260 & 520 & 4 & 1 & 7 & 2 \\ \hline
1076 & 18,3 & 273 & 546 & 4 & 1,5 & 7 & 2,5 \\ \hline
1080 & 19,2 & 286 & 572 & 5 & 1,5 & 8 & 2,5 \\ \hline
1084 & 20 & 299 & 598 & 5 & 1,5 & 8 & 2,5 \\ \hline
1088 & 20,9 & 312 & 624 & 5 & 2 & 8 & 3 \\ \hline
1092 & 21,8 & 325 & 650 & 5 & 2 & 9 & 3 \\ \hline
1096 & 22,7 & 338 & 676 & 6 & 2 & 9 & 3 \\ \hline
1100 & 23,5 & 351 & 702 & 6 & 2 & 10 & 3 \\ \hline
\end{tabular}
\end{table}

OBS:

- A situação do fermento é caracterizada por diversos fatores como armazenagem, validade e manejo.

- A tabela contém algumas aproximações


Além disso, o mosto deve passar por um processo de oxigenação para que o processo de fermentação ocorra de maneira eficiente através da multiplicação das leveduras num primeiro momento. Geralmente isso é realizado agitando o mosto a fim de formar bolhas de oxigênio.

A levedura é então dissolvida em água (respeitando a temperatura indicada para cada tipo de levedura) para que depois possa ser adicionada ao mosto.


\subsubsection{Fermentação}

A próxima etapa consiste na adição da levedura onde o processo de fermentação se inicia. Após a adição, é necessário anexar uma válvula air-lock ao mosto para que não haja entrada de oxigênio ou agentes contaminantes, mas que haja saída de gás carbônico produzido pelas leveduras.

Nesta etapa é muito importante realizar o controle de temperatura para que a fermentação obtenha os resultados esperados.

A fermentação ocorre em 3 fases:

\subsubsection{Respiração}

Também conhecido como “Lag time”. É nessa fase que o fermento se reproduz até atingir sua massa crítica e passar para a segunda fase, a fermentação. Essa fase deve ser minimizada para que seja possível evitar a invasão por bactérias que possam prejudicar o processo. Uma boa sanitização e inoculação (hidratação/ativação) correta da levedura são partes essências.

\subsubsection{Fermentação}
Nesta fase ocorre a transformação de açúcares presentes no mosto em CO2 e álcool. Para saber se sua fermentação está ocorrendo basta conferir se há bolhas no air-lock.

\subsubsection{Sedimentação}

Acontece quando o nível de açúcar começa cair e o fermento começa a decantar para a parte inferior do fermentador. O tempo necessário para que a fermentação acabe depende do tipo de levedura, da temperatura utilizada.

\subsection{Maturação}

Após a fermentação principal, a agora então chamada cerveja, que ainda possui uma suspensão de leveduras e uma parte de material fermentescível, passa por uma fermentação secundária, chamada maturação. Geralmente esta etapa do processo é realizada sob temperaturas mais baixas, próximas à zero, e pode levar até 21 dias, o que contribui para clarificação da cerveja e melhora de seu sabor e aroma.

Nesta etapa o liquido é separado do fermento, que neste ponto está ao fundo do fermentador.

\subsection{Priming}
Ao final da maturação, separa-se o liquido fermentado dos resíduos de fermento decantados no reservatório transferindo-o para outro reservatório. Com isso, a mistura principal estará pronta para receber o Priming, que consiste numa mistura de agua com algum tipo de açúcar, que pode ser glicose, sacarose, dextrose, mel entre outros tipos de açúcar. Comumente utiliza-se açúcar invertido, produzido utilizando açúcar cristal ou refinado.

Este processo serve para que ocorra uma refermentação da cerveja em seu vaso final, para que a levedura possa produzir a partir deste açúcar mais álcool (nesta fase em quantidade insignificante para a medição de teor alcoólico da cerveja) e o gás carbônico, que vai gaseificar a cerveja.

Para que seja preparado necessário saber quantos litros de cerveja serão produzidos, normalmente o Priming é produzido fervendo-se de aproximadamente 6g de açúcar (cristal ou refinado) em 150ml de água para cada litro de solução de cerveja. Esta mistura é fervida até que fique vire uma calda homogênea e, após isso, esta mistura formada é adicionada no reservatório da cerveja e misturada.

\subsection{Envase}

Após o processo de Priming, a cerveja está pronta para ser envasada, para isso utiliza-se garrafas de vidro com tampas de metal, ou a cerveja é envasada num barril, em ambos os casos para a refermentação. Porém antes que a passagem da cerveja seja feita do reservatório onde aconteceu o Priming para seu recipiente final (garrafa ou barril), é necessário que este seja devidamente higienizado e sanitizado.

\subsubsection{Higienização}

No caso das garrafas, a higienização é feita com água e um escova simples que chegue ao fundo da garrafa para que sejam retiradas qualquer tipo de detrito, tanto do fundo quanto das bordas. No caso de um barril, o processo é o mesmo, porém utiliza-se uma escova maior para higienizar as laterais e o fundo.

\subsubsection{Sanitização}

Normalmente a sanitização dos recipientes é feita com ácido peracético (C2H4O3), diluído em água a uma proporção de 0,2\% de ácido peracético por litro de água. Essa proporção é normalmente alcançada utilizando-se ácido peracético em pó, comercializado como desinfetante para alimentos, superfícies e utensílios em geral.

Um exemplo é o PAC 200, um desinfetante em pó à base de ácido peracético bastante utilizado por cervejeiros artesãos. Para que chegue à concentração indicada de 0,2\% de ácido por litro de água, devem ser diluídos 1g de produto para cada litro de água. Essa mistura consegue sanitizar um utensílio removendo até 99,9\% das bactérias em cerca de 20 minutos de contato. Neste tipo de solução o utensílio deve ser submerso ou preenchido com a solução não necessitando um enxague posterior, somente que seja escorrido todo o líquido.

\subsection{Refermentação}
Após o envasamento da cerveja ela deve ser deixada por aproximadamente 10 a 15 dias fechada no recipiente final para que ocorra a refermentação, que nada mais é que um novo processo de fermentação, só que agora em temperatura ambiente, da cerveja para que sejam liberados álcool (em quantidade insignificante para a medição de teor alcoólico da cerveja) e gás carbônico (CO2) que será responsável pela carbonatação (ou gaseificação) da cerveja no recipiente.
