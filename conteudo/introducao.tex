\section{Introdução}
    \subsection{Descrição do Problema}

    \subsection{Justificativa}
    Brassagem, recirculação, lavagem, fervura, resfriamento, fermentação e maturação, estes são os principais processos envolvidos na fabricação da cerveja. Sabe-se que esses procedimentos são simples, porém requerem atenção e paciência quando feitos manualmente. Alguns aspectos podem gerar um resultado diferente do esperado, um exemplo disso é a temperatura que deve ser respeitada rigorosamente de acordo com a receita de cada tipo de cerveja. Para obter uma menor chance de contaminação, controle mais preciso da temperatura, menor possibilidade de acidentes e qualidade no resultado final é que se pensa na automatização deste processo (Telles, 2014).  Percebe-se que várias áreas estão envolvidas nestas etapas e assim é possível integrar as engenharias e aplicar as habilidades e os conceitos adquiridos no decorrer do curso para obter o produto final proposto.

    \subsection{Objetivos}

    \begin{itemize}
        \item \textbf{Objetivo Geral:} Projetar, desenvolver e confeccionar uma mini fábrica de cerveja.
        \item \textbf{Objetivos Específicos:}
        \begin{itemize}
            \item Estudar e entender os processos de fabricação de cerveja;
            \item Tornar estes processos automatizados ou semi-automatizados;
            \item Integrar as engenharias;
            \item Desenvolver a habilidade dos participantes de trabalhar em conjunto.
        \end{itemize}
    \end{itemize}
