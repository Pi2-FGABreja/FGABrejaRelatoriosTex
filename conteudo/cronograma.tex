\section{Cronograma e Metodologia}

A equipe de projeto possui 14 membros de 4 engenharias distintas, representando um desafio estratégico para definição de um escopo que integre as 4 áreas e seja corretamente encaminhado, de forma que viabilize a qualidade do trabalho, do produto e não sobrecarregue nenhum dos integrantes em suas áreas de atuação.

Para cumprir esse propósito, o grupo buscou no planejamento de projeto definido pelo PMBOK boas práticas para conduzir o trabalho, e nos princípios ágeis a coordenação da equipe com objetivos claros de circular o conhecimento entre as áreas. Foram feitos alguns dos planos de gerenciamento adotados pelo PMBOK, que se adequavam ao perfil da equipe e no contexto do projeto.

\subsection{Gerenciamento da Comunicação}

O plano de comunicação foi discutido na primeira reunião oficial da equipe. Foi estipulada uma reunião semanal de alinhamento nas quartas-feiras, das 16h às 18h, horário de aula da disciplina Projeto Integrador 2. Nessa reunião as decisões seriam tomadas e foi adotada uma técnica para fazer o conhecimento entre os grupos circular. Nesse espaço de tempo, cada membro do projeto pode explicar o que produziu e indicar para os demais suas restrições.

Para a comunicação casual, e diária, foi escolhida a princípio o whats app, porém, após a primeira semana de trabalho, houveram algumas necessidades em relação a divisão de grupos de trabalho e integração com outras ferramentas utilizadas pelo grupo, o que nos levou a escolha da ferramenta de mensagens instantâneas Slack, que oferece, além da comunicação em tempo real, um ambiente propício para o trabalho em equipe, colaboração e integração com diversas ferramentas tais como Trello e Google Drive que também são utilizadas pelo grupo.

\subsection{Gerenciamento do Tempo}

O tempo limite para término do projeto e entrega da solução é o final do semestre e da disciplina ao qual o projeto está inserido. Neste meio tempo, existem duas entregas intermediárias que serão encaradas como marcos do projeto, totalizando assim 3 entregas. As atividades e distribuição de trabalho serão baseados nos marcos definidos, ainda que eles não possuam data definida, o grupo trabalha com as estimativas fornecidas pelos professores/orientadores da disciplina.

Os seguintes processos serão desenvolvidos ao longo do semestre, com relação ao tempo:

\begin{itemize}
    \item \textbf{Definir as Atividades:} Identificar e documentar as ações específicas a serem realizadas para produzir as entregas do projeto.
    \item \textbf{Sequenciar as Atividades:} Identificar e documentar os relacionamentos entre as atividades do projeto.
    \item \textbf{Estimar as Durações das Atividades:} Estimar o número de períodos de trabalho que serão necessários para terminar atividades específicas com os recursos estimados.
    \item \textbf{Desenvolver o Cronograma:} Analisar as sequências das atividades, suas durações, recursos necessários e restrições do cronograma visando criar o modelo do cronograma do projeto.
    \item \textbf{Controlar o Cronograma:} Monitorar o andamento das atividades do projeto para atualização do seu progresso e gerenciamento das mudanças feitas na linha de base do cronograma para realizar o planejado.
\end{itemize}

\subsection{Gerenciamento do Escopo}

A definição do escopo será definida pelos membros da equipe de projeto, sob a orientação dos professores da disciplina. As decisões estratégicas serão tomadas em equipe, nas reuniões de alinhamento semanais. O processo para a definição do escopo segue os seguintes passos:

\begin{itemize}
    \item Pesquisa conceitual
    \item Estudo de viabilidade
    \item Alinhamento da proposta com o planejamento estratégico
    \item Votação
\end{itemize}

\subsection{Gerenciamento da Qualidade}
Existe uma complexidade inerente à este projeto que se revela no caráter integrador das diferentes engenharias. A condução do projeto pode se perder nas dificuldades e restrições técnicas de cada equipe, portanto, é importante que se observem alguns princípios de gestão, a fim de que o trabalho possa fluir de forma natural dentro do processo de desenvolvimento.

Para assegurar a qualidade do processo de desenvolvimento com expectativas que o produto final seja impactado pela qualidade do processo, se trouxe das metodologias ágeis algumas práticas de gestão. Os princípios abordados são os de integração da equipe de projeto.

Desta forma, foi estipulado que em todas as reuniões semanais de alinhamento cada membro do grupo teria uma atividade e fosse avaliado pelo grupo inteiro na próxima semana, indicando o valor agregado da sua participação dentro do projeto.

Cada membro do grupo deve dar uma nota a si mesmo, variando de 0 a 5, de acordo com o seu desempenho naquela semana do projeto.

Os membros do grupo que trabalharam com aquele participante podem intervir, avaliando o colega pelo que acreditam que tenha sido a contribuição dele no projeto aquela semana.

Por ultimo, algum membro do projeto pode intervir, indicando uma nota para aquele membro do projeto, caso julgue necessário.

