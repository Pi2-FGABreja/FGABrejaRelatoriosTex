%\documentclass[journal,compsoc]{IEEEtran}\newcommand{\journal}{true}
%\documentclass[conference]{IEEEtran}\newcommand{\journal}{false}
%\documentclass[a4paper]{coursepaper}
%\documentclass[12pt,answers]{exam}
%\documentclass[a4paper,11pt]{article}
%\documentclass[a4paper,11pt]{book}
\documentclass[a4paper,11pt]{report}
\usepackage[a4paper,margin={2cm,2cm}]{geometry}
\usepackage{lmodern}
\usepackage{marvosym}
\usepackage[utf8x]{inputenc}
\usepackage[english,brazil]{babel}
\usepackage[T1]{fontenc}
\usepackage{indentfirst}
\usepackage{lipsum}
\usepackage{listings}
\usepackage[nocompress]{cite}
\usepackage{multicol}
\usepackage[usenames,dvipsnames,table]{xcolor}
\usepackage{lastpage}
\usepackage{makeidx}
\usepackage{longtable}
\usepackage{datetime}

\settimeformat{xxivtime}
%\usepackage[num,abnt-year-extra-label=yes,abnt-full-initials=no,abnt-emphasize=bf,abnt-and-type=e,abnt-url-package=url,abnt-repeated-author-omit=yes]{abntcite}%{abnt-alf}
%\usepackage[num]{abntcite}
\lstset{%
language=C++,							%linguagem
numbers=left,						%posição dos números
stepnumber=1,						%frequencia de aparição dos números
numbersep=0.5pt,
basewidth={0.6em,0.45em},
fontadjust=true,
mathescape=true,
tabsize=4,
commentstyle=\color{blue},
literate={á}{{\'a}}1 {à}{{\`a}}1 {ã}{{\~a}}1 {é}{{\'e}}1 {É}{{\'E}}1 {ê}{{\^e}}1 {õ}{{\~o}}1 {í}{{\'i}}1 {ó}{{\'o}}1 {ú}{{\'u}}1 {ç}{{\c c}}1 {³}{{$^3$}}1 {Ω}{{$\Omega$}}1,
breaklines=true,
showstringspaces=false,
stringstyle=\color{cyan},
basicstyle=\small\ttfamily}

\usepackage[pdftex]{graphicx}
\pdfcompresslevel=1

\usepackage[center]{caption}
\usepackage{float}
\usepackage{fixltx2e}
\usepackage{longtable}
\usepackage{tabularx}
%\usepackage[table]{xcolor}
\usepackage{tikz-uml}


% *** GRAPHICS RELATED PACKAGES ***
%

\usepackage[
pdfauthor={Matheus Fernandes},
pdftitle={Automa\c{c}\~ao do Processo de Fabrica\c{c}\~ao de Cerveja},
pdfsubject={FGA Breja},
pdfkeywords={cerveja},
backref=true,
draft=false,
%pdfstartview=fitR,
bookmarks=true,
bookmarksopen=true,
colorlinks=true,
linkcolor=black,
urlcolor=black,
citecolor=black,%blue
pdftex,
bookmarks=true,
linktocpage=true,   % makes the page number as hyperlink in table of content
hyperindex=true,
unicode=false
]{hyperref}
\usepackage[all]{hypcap}

%\usepackage{glossaries}
%\usepackage[toc]{glossaries}
\usepackage[%
% acronym % use acronym functionality
%,section = section % use sections for all glossary lists
,nonumberlist
% ,xindy={language=portuguese}
]{glossaries}

% \usepackage[xindy={language=portuguese},nonumberlist=true]{glossaries}

\makeglossaries


%-------------------------------------------------------------------------------

% *** GRAPHICS RELATED PACKAGES ***
%
\usepackage{tikz}
\usetikzlibrary{shapes,arrows}
\usetikzlibrary{calc,decorations.pathmorphing,patterns}
\pgfdeclaredecoration{penciline}{initial}{
    \state{initial}[width=+\pgfdecoratedinputsegmentremainingdistance,
    auto corner on length=1mm,]{
        \pgfpathcurveto%
        {% From
            \pgfqpoint{\pgfdecoratedinputsegmentremainingdistance}
                      {\pgfdecorationsegmentamplitude}
        }
        {%  Control 1
        \pgfmathrand
        \pgfpointadd{\pgfqpoint{\pgfdecoratedinputsegmentremainingdistance}{0pt}}
                    {\pgfqpoint{-\pgfdecorationsegmentaspect
                     \pgfdecoratedinputsegmentremainingdistance}%
                               {\pgfmathresult\pgfdecorationsegmentamplitude}
                    }
        }
        {%TO
        \pgfpointadd{\pgfpointdecoratedinputsegmentlast}{\pgfpoint{1pt}{1pt}}
        }
    }
    \state{final}{}
}


% \usepackage{tikzgraphicx}

\tikzstyle{cloud} = [draw, ellipse,fill=blue!20, node distance=3cm,
    minimum height=3em]
\tikzstyle{estado} = [draw, ellipse,fill=blue!20, node distance=3cm,align=center]
\tikzstyle{teste} = [draw, diamond,fill=green!20, node distance=3cm,align=left]
%    minimum height=2em]
\tikzstyle{ciclo} =[draw, rectangle,fill=blue!20, node distance=3cm,align=center,decorate,thick,minimum height=3em]
\tikzstyle{phanton} = []
\tikzstyle{line} = [->,right] %[draw, -latex']
\tikzstyle{retorno} = [loop above]
\tikzstyle{arrow} = [bend left,->]
\tikzset{
    %Define standard arrow tip
    >=stealth',
    %Define style for boxes
    punkt/.style={
           rectangle,
           rounded corners,
           draw=black, very thick,
           text width=6.5em,
           minimum height=2em,
           text centered},
    % Define arrow style
    pil/.style={
           ->,
           thick,
           shorten <=2pt,
           shorten >=2pt,}
}

\pdfcompresslevel=9

%-------------------------------------------------------------------------------

\newcounter{alphasect}
\def\alphainsection{0}

\let\oldsection=\section
\def\section{%
  \ifnum\alphainsection=1%
    \addtocounter{alphasect}{1}
  \fi%
\oldsection}%

\renewcommand\thesection{%
  \ifnum\alphainsection=1%
    \Alph{alphasect}
  \else%
    \arabic{section}
  \fi%
}%

\newenvironment{alphasection}{%
  \ifnum\alphainsection=1%
    \errhelp={Let other blocks end at the beginning of the next block.}
    \errmessage{Nested Alpha section not allowed}
  \fi%
  \setcounter{alphasect}{0}
  \def\alphainsection{1}
}{%
  \setcounter{alphasect}{0}
  \def\alphainsection{0}
}%


%\makeglossaries

\input glossary

\makeindex
\begin{document}

\begin{titlepage}

\newcommand{\HRule}{\rule{\linewidth}{0.5mm}} % Defines a new command for the horizontal lines, change thickness here

\center % Center everything on the page

%----------------------------------------------------------------------------------------
%   HEADING SECTIONS
%----------------------------------------------------------------------------------------

\textsc{\LARGE Universidade de Brasília}\\[1.5cm] % Name of your university/college
\textsc{\Large Projeto Integrador 2 - 2\textdegree/2015}\\[0.5cm] % Major heading such as course name
\textsc{\large Grupo X}\\[0.5cm] % Minor heading such as course title

%----------------------------------------------------------------------------------------
%   TITLE SECTION
%----------------------------------------------------------------------------------------

\HRule \\[0.4cm]
{ \huge \bfseries Automação do Processo de Fabricação de Cerveja}\\[0.4cm] % Title of your document
\HRule \\[1.5cm]

%----------------------------------------------------------------------------------------
%   AUTHOR SECTION
%----------------------------------------------------------------------------------------

\begin{minipage}{0.4\textwidth}
\end{minipage}\\[4cm]

% If you don't want a supervisor, uncomment the two lines below and remove the section above
%\Large \emph{Author:}\\
%John \textsc{Smith}\\[3cm] % Your name

%----------------------------------------------------------------------------------------
%   DATE SECTION
%----------------------------------------------------------------------------------------

{\large \today}\\[3cm] % Date, change the \today to a set date if you want to be precise

%----------------------------------------------------------------------------------------
%   LOGO SECTION
%----------------------------------------------------------------------------------------

%\includegraphics{Logo}\\[1cm] % Include a department/university logo - this will require the graphicx package

%----------------------------------------------------------------------------------------

\vfill % Fill the rest of the page with whitespace

\end{titlepage}

\pagenumbering{roman}

\selectlanguage{brazil}		%voltando para o português.

\tableofcontents			% Sumario

%\phantomsection
\listoffigures
\addcontentsline{toc}{section}{Lista de Figuras}

%\phantomsection
\listoftables
\addcontentsline{toc}{section}{Lista de Tabelas}


\setcounter{page}{1}
\pagenumbering{arabic}
%	\thispagestyle{empty}
%	\IEEEdisplaynotcompsoctitleabstractindextext
%	\IEEEpeerreviewmaketitle

%\onecolumn		%Para texto em uma coluna
%\twocolumn		%Para texto em duas colunas
%\begin{multicols}{2}
%\begin{alphasection}

\newpage
\input import_conteudos



	\rowcolors{0}{white}{white}
%\end{alphasection}
%\end{multicols}
%\onecolumn
%\printglossaries
%\glsdescwidth


\onecolumn
\newpage
\providetranslation{Notation (glossaries)}{Notação}
\providetranslation{Description (glossaries)}{Descrição}
\phantomsection
\addcontentsline{toc}{section}{\glossaryname}
\setlength{\glsdescwidth}{0.48\textwidth}%
% \printglossary[style=superheaderborder]
\glossarystyle{altlisthypergroup}
\printglossaries.
%superheaderborder
%altlisthypergroup
\newpage
\phantomsection
\addcontentsline{toc}{section}{Referências Bibliográficas}
	\bibliographystyle{abnt-alf}%{IEEEtran}%{abnt-num}%{ieeetr}%{abnt-alf}
	\bibliography{bibliography}		% expects file "bibliography.bib"

\end{document}
